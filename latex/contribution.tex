%
 Viral pathogens such as influenza, coronavirus, and HIV remain a high
 source of global mortality and morbidity, beause their rapid rate of
 mutation presents serious challenges to vaccine design.
%
 Many anti-viral vaccines are currently in development, but to be
 effective against future strains, they must raise host immunity to
 structurally conserved regions of the virus. Unfortunately, viral
 adaptation conceals such conserved epitopes from the immune system and
 guides host immunity towards highly variable, distracting epitopes.
 Success in developing vaccines against rapidly mutating viruses depends
 critically on understanding and learning how to overcome this distraction.
%
 We have developed and validated a computer model of adaptive
 immunity, in which we simulate the development of high-affinity
 antibodies to antigenic epitopes with different properties. 
%
 We find that the mode of antibody binding to antigens can explain
 experimental results that show that the immune system is easily
 directed from certain conserved epitopes toward variable ones.  We
 performed additional simulations to explore the influence of antigen
 concentration on the competition between different antibodies, and
 whether raised immunity to conserved (but concealed) epitopes could be
 long lasting. Our results are expected to inform design of universal vaccines.
